% !TEX root = ../vr_st.tex

\begin{abstract}
	Persistent homology is a central tool for the study of filtered spaces in both applied and theoretical topology.
	However, as with homology in the context of unfiltered homotopy types, persistent homology fails to capture much of the inherent structure of their filtered counterparts.
	In this work, we establish the foundations of the theory of persistent cohomology operations, proving their Gromov–Hausdorff stability and deriving formulas for wedge sums and products of metric spaces.
	By leveraging these results, along with generalizations of Gromov’s filling radius, we construct examples of Riemannian pseudomanifolds for which the metric estimates derived from these new invariants are strictly sharper than those obtained from persistent homology, thereby underscoring their enhanced discriminatory power.
\end{abstract}