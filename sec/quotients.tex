% !TEX root = ../vr_st.tex

\subsection{General quotients}

\subsubsection{}

%An action of a group $G$ on a set $X$ is a function $G\times X\to X$ such that $g\cdot (h\cdot x)=(gh)\cdot x$, and $e\cdot x = x$ for all $x\in X$ and $g,h\in G$, where $e$ is the identity element of $G$.

Let $G$ be a group acting on a metric space $\cX$ and $\cX_G$ be the quotient of $\cX$ under the group action $G$.
Elements in $\cX_G$ will be denoted as $[x]$ for $x \in \cX$.
We say the action of $G$ is \defn{proper} if, for every $x \in \cX$, there is some $r>0$ such that $\{g \mid g\cdot B(x,r) \cap B(x,r) = \emptyset\}$ is finite.
We say $G$ \defn{acts by isometries} on $\cX$, if the map $g \colon \cX \to \cX$ is an isometry for very $g \in G$.

Let $G$ be a group acting properly and by isometries on a metric space $\cX$.
Then the \defn{quotient metric}
\[
d_{\cX_G}\big([x], [x']\big) \defeq \inf_{g\in G} d_\cX(x, g\cdot x')
\]
is a well-defined metric on $\cX_G$.

\subsubsection{}

Let $r>0$.
The action of a group $G$ on $\cX$ is an \defn{$r^*$-diameter action} if for any non-negative integer $k$, $\diam_{\cX_G}\{[x_0],\dots,[x_k]\}<r$ implies that there exist unique choice of $g_i$'s for $1\leq i\leq k$ such that $\diam_{\cX}\{x_0,g_1x_1,\dots,g_kx_k\}<r$.
\anibal{Please finish this. Why \(r \neq 0\)? Whats the quantifier: for all or there is?
	\ling{We need to be careful with the citation here: (1) cite \cite{adams2022metric} for the original definition; (2) cite Liam Barham for the fixed definition. We should ask Liam Barham for permission to cite his unpublished work.}}

\subsubsection{}\label{subsub:h}

The action of $G$ on $\cX$ induces a group action of $G$ on the Vietoris--Rips complexes $\VR_r(\cX)$ defined by
\[
g \cdot \sum \lambda_i x_i \defeq \sum \lambda_i (g\cdot x_i).
\]
%We begin by examining the relationship between the Vietoris--Rips filtration of a quotient space and the Vietoris--Rips filtration induced by the quotient.
The following proposition is proved in \cite[Proposition 3.5]{adams2022metric}, we include a proof sketch below for clarity.\anibal{Is the inverse map, the \(\tilde h\), ever used? I think the proof sketch does not add clarity. I would remove it.}

\medskip\lemma
Let $G$ acts properly and by isometry on $\cX$, and the action of $G$ on $\cX$ is a $r^*$-diameter action for $r>0$.
Then, the map
\begin{align*}
	h \colon \VR_r(\cX_G) &\to (\VR_r(\cX))_G \\
	\textstyle\sum_i\lambda_i [x_i] &\mapsto \textstyle \big[\sum_i\lambda_i x_i\big]
\end{align*}
is an isomorphism of simplicial complexes.\anibal{\(r*\) does not imply proper?}

%\begin{proof}
%	Because $G$ acts by isometry, we have a well-defined map
%	\[
%	\tilde{h} \colon \VR_r\cX \to \VR_r(\cX_G)
%	\text{ with }
%	\sum_{i=1}^k \lambda_i x_i \mapsto \sum_{i=1}^k \lambda_i [x_i],
%	\]
%	Because two points in the same orbit of the $G$ action always have the same image under $\tilde{h}$, it induces a map $\tilde{h}_G \colon (\VR_r\cX)_G \to \VR_r(\cX_G)$.
%
%	Moreover, $\tilde{h}_G$ is an isomorphism, following from the fact that the action of $G$ on $\cX$ is an $r^*$-diameter action; see \cite[Proposition 3.5]{adams2022metric} for further details.
%	Therefore, $h$, the inverse of $\tilde{h}_G$, is also an isomorphism.
%\end{proof}

\subsection{Quotients of spheres}

\subsubsection{}\label{subsub:f}

Consider the composition
\[
f_r \colon \VR_r(\bS^n) \to \R^{n+1} \setminus \set{0} \to \bS^n,
\]
where the first map sends a formal linear combination $\sum_i\lambda_i x_i$ in the Vietoris--Rips complex $\VR_r(\bS^n)$ to the point $\sum_i\lambda_i x_i \in \bbR^{n+1}$
%where $x_i \in \bS^n$ and $\lambda_i \in [0,1]$ satisfying $\sum_i\lambda_i = 1$,
and the second map is the radial projection.

\medskip\lemma
The map $f_r$ is well defined if $r < \pi$ and it is a homotopy equivalence when $0 < r < \zeta_n$.\anibal{why is r=0 excluded?}
\footnote{For $0 < r < \zeta_n$, the Vietoris--Rips complex $\VR_r(\bS^n)$ is indeed homotopic to $\bS^n$, as shown in \cite[Theorem 7.1]{lim2020vietoris}.\anibal{This comment should be something like: in ... a different map was used to show that ... (I guess is the inclusion) I am assuming that is the case, otherwise there would be no need for this proof.}}

\begin{proof}
	To prove the statement, we use an intermediate construction called the Vietoris--Rips thickening, introduced in \cite{adamaszek2018metric} as a metric space analogue of the Vietoris--Rips complexes.

	The Vietoris--Rips thickening $\VR_r^m\cX$ of a metric space $\cX$ at a given scale $r$ is defined as the set of convex linear combinations of the Dirac probability measures $\delta_{x}$ on points $x \in X$, equipped with the $1$-Wasserstein metric.
	There is a set bijection from $g \colon \VR_r\cX \to \VR_r^m\cX$, given by $\sum_i \lambda_i x_i \mapsto \sum_i \lambda_i \delta_{x_i}.$
	%As this object is not the center of the discussion here, we omit the details and only state the related results here.
	It is shown in \cite[Theorem 1]{gillespie2024vietoris} that (for open Vietoris--Rips complexes) the natural bijection $g$ is a weak homotopy equivalence.\anibal{What is the word open doing? How does it relate to our definition. If our definition is open, the the word should be removed.}

	In the case of $\cX = \bS^n$, the map $f$ is the composition of two maps
	\[
	\VR_r(\bS^n) \xrightarrow{g} \VR_r^m(\bS^n) \xrightarrow{f \circ g^{-1}} \bS^n.
	\]
	According to \cite[Proposition 5.3]{adamaszek2018metric}, when $0<r<\zeta_n$, $f \circ g^{-1}$ is a homotopy equivalence, implying that $f = (f \circ g^{-1}) \circ g$ is a weak homotopy equivalence.\anibal{should be \(f_r\)}
\end{proof}

\subsubsection{}\label{subsub:rho}

A group action on \(\bS^n\) is said to be \defn{\(r\)-spherical} if it commutes with \(f_r\) as defined in \cref{subsub:f}.
%Let $G$ be a group action on $\bS^n$ that commutes with $f$.
%Let $f_G \colon \VR_r(\bS^n)_G \to \bS^n_G$ be the induced map of $f$.

\medskip\lemma
For any $0 < r < \zeta_n$, the following composition is a weak homotopy equivalence
\[
\rho \colon \VR_r(\bS^n)_G \xra{(f_r)_G} ? \xra{h} \bS^n_G.\anibal{the domain and target of \(h\) seem wrong}
\]
\begin{proof}
	By \cref{subsub:f}, the map $f$ is a weak homotopy equivalence when $0<r<\zeta_n$.
	Then, because $G$ is a group action on $\bS^n$ preserved by $f$, the induced map $f_G$ is also a weak homotopy equivalence.
	Furthermore, as the map $h$ (defined in \cref{subsub:h}) is a homomorphism, its composition with $f_G$ is a weak homotopy equivalence.\anibal{Homo? not iso?}
\end{proof}

\subsubsection{}

\anibal{I would only consider the first square. Furthermore, I would make its commutativity a property of the action that will be later verified for real projective spaces.}

For $0 < t\leq s \leq s'$, consider the following diagram of topological spaces:
\begin{equation}\label{d:fundamental_bars_diagram}
	\begin{tikzcd}
		\bS^{n-1}_G
		\ar[d, hook,"{\iota}" left]
		&
		\VR_t(\bS^{n-1}_G)
		\ar[d, hook,"\iota_t"]
		\ar[l, "\ \rho^{n-1}" above]
		\ar[r, hook, "v^{n-1}"]
		&
		\VR_{s}(\bS^{n-1}_G)
		\ar[d, hook]
		\\
		\bS^{n}_G
		&
		\VR_t(\bS^{n}_G)
		\ar[l, "\rho^n" above]
		\ar[r, hook, "v^{n}"]
		&
		\VR_{s'}(\bS^{n}_G),
	\end{tikzcd}
\end{equation}
where $\rho^n$ and $\rho^{n-1}$ are the weak homotopy equivalence defined in \cref{subsub:rho}, the horizontal inclusions $v^{n-1}$ and $v^n$ are induced by the Vietoris--Rips filtration, and the vertical maps are induced by the inclusions $\iota \colon \bS^{n-1}_G \hookrightarrow \bS^{n}_G$.
\anibal{The existance of these inclusions is not for free. I think one needs to assume that the equatorial inclusions are equivariant wrt a group acting on every sphere, at least, two consecutives, before this makes sense}

\medskip\lemma
Diagram \eqref{d:fundamental_bars_diagram} commutes.

\begin{proof}
	The commutativity of the right-hand-side square is straightforward.
	For the left-hand-side square, take any $y = \sum_{i=1}^k \lambda_i [x_i] \in \VR_t(\bS^{n-1}_G)$ and verify that
	\begin{center}
		$(\iota \circ \rho^{n-1})(y)
		=\iota(f^{n-1}_G([\sum_i \lambda_i x_i]))
		=\iota([f^{n-1}(\sum_i \lambda_i x_i)])
		=[\pi^{n-1}(\sum_i \lambda_i x_i)]
		$
	\end{center}
	as an element in $\bS^n_G$, and
	\begin{center}
		$(\rho^{n} \circ \iota_t)(y) = \rho^{n}(y) = f^{n}_G([\sum_i \lambda_i x_i]) = [f^{n}(\sum_{i=1}^k \lambda_i x_i)] = [\pi^{n}(\sum_i \lambda_i x_i)].
		$
	\end{center}
	Because $\pi^{n}$ restricted to $\bS^{n-1}_G$ is equal to $\pi^{n-1}$, we conclude that $(\iota \circ \rho^{n-1})(y) = (\rho^n \circ \iota_t)(y)$ for any $y$.
	Thus, the diagram commutes.\anibal{Please use `big' for parenthesis that are nested.}
\end{proof}

\subsubsection{}\label{subsub:foundamental_bar_rpn_lemma}

Let $\delta_n=2\mathrm{Fillrad}(\bS^n_G)$.
%, which is clearly bounded below by $\alpha_n$ the first critical value of $\VR(\bS^n_G)$.
For any degree $1 \leq \degp \leq n$, let $\beta_{\degp}^{n}$ be the first critical value of the $\degp^\th$ homology of $\VR(\bS^n_G)$.
It follows from \cref{subsub:beta v.s. fillrad} that $(\alpha \leq) \beta_{n}^{n} \leq \delta_n$.
We represent these values using the following lower triangular matrix:
\[
\begin{pmatrix}
	\beta_{1}^{1}\leq \delta_1 & & &&\\
	\beta_1^2 & \beta_{2}^{2} \leq \delta_2 & &&\\
	\beta_1^3 & \beta_{2}^{3} & \beta_{3}^{3} \leq \delta_3 &&\\
	\dots & \dots & \dots & \dots &\\
	\beta_1^n & \beta_2^n & \beta_3^n & \dots & \beta_n^n \leq \delta_n
\end{pmatrix}
\]
In the lemma below, we show that all elements in the above matrix are bounded above by $\delta_n$, if $\{\delta_1, \dots, \delta_n\}$ is monotonically non-decreasing.

\medskip\lemma
If $\delta_1 \leq \dots \leq \delta_n$, then $\beta_{\degp}^{n} \leq \delta_n$ for any $1 \leq m \leq n$.
\anibal{DO you need the inequalities on the deltas? I guess this can be stated using the \(\max\)}

\begin{proof}
	We will use an induction argument on $n$.
	When $n = 1$, $\beta_{1}^{1} \leq \delta_1$ holds.
	%Because $\delta_1 \leq \alpha_1$, $\rH_1(\VR\bS^1_G)$ retains the same isomorphism type before $\delta_1$, implying that $\beta_{1, 1} \geq \delta_1$.
	%Thus, $\beta_{1, 1} =\delta_1$.

	Assume the statement holds for $\bS^{n-1}_G$, that is, $\beta_{\degp}^{n-1} \leq \delta_{n-1}$ for any $1\leq m \leq n-1$.
	Since $\delta_{n-1} \leq \delta_n$, applying the $\degp^\th$ homology functor to diagram \eqref{d:fundamental_bars_diagram}, we obtain the following commutative diagram of vector spaces:
	for $r,\epsilon>0$ small,
	\[
	\begin{tikzcd}
		\rH_\degp(\bS^{n-1}_G)
		\ar[d, "\cong" left]
		&
		\rH_\degp(\VR_r\bS^{n-1}_G)
		\ar[d, "\rH_\degp(\iota_r)" left, "\cong" right, myred]
		\ar[l, "\cong" above]
		\ar[rr, "\rH_\degp(v^{n-1})", myred]
		&
		&
		\rH_\degp(\VR_{\delta_{n-1}+\epsilon}\bS^{n-1}_G)
		\ar[d]
		\\
		\rH_\degp(\bS^{n}_G)
		&
		\rH_\degp(\VR_r \bS^{n}_G)
		\ar[l, "\cong"]
		\ar[rr, "\rH_\degp(v^n)" , myred]
		&
		&
		\rH_\degp(\VR_{\delta_n+\epsilon}\bS^{n}_G).
	\end{tikzcd}
	\]
	Here, $\rH_\degp(f)$ for some map $f$ between topological spaces denotes the induced map on the $\degp^\th$ homology.

	Let $\sigma$ be a representative cycle for the bar $(0,\, \beta_{\degp}^{n-1})$ in $\VR \bS^{n-1}_G$.
	Commutativity of the left-hand-side square implies that $\rH_\degp(\iota_r)$ is an isomorphism, implying that $\rH_\degp(\iota_r)(\sigma)$ is born at $0$.
	It follows from $\beta_{\degp}^{n-1} \leq \delta_{n-1}$ that $\rH_\degp(v^{n-1})(\sigma) = 0$.
	Using the right-hand-side square's commutativity, we deduce $(\rH_\degp(v^n) \circ \rH_\degp(\iota_r))(\sigma)=0$, which means $\rH_\degp(\iota_r)(\sigma)$ dies after $\delta_n+\epsilon$.
	This holds for any $\epsilon>0$, so we can conclude $\beta_{\degp}^{n} \leq \delta_n$.

	For the case when $\degp=n$, we also have $\beta_{n}^{n} \leq \delta_n$.
	This completes the proof.
\end{proof}