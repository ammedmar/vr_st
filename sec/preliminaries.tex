% !TEX root = ../vr_st.tex

\section{Background}

We recall some of the basic notions of the theory of persistence.

\subsection{Persistence modules}

Throughout the paper we think of the set of real numbers $\R$ as a category via its poset structure.
We also fix a field $\kk$ and write $\Vec$ for the category of vector spaces over $\kk$.

\subsubsection{} A \defn{persistence module} $V$ is a functor from $\R$ to $\Vec$.
Explicitly, it consists of a vector space $V_r$ for each $r \in \R$ and a linear map $V_{s,t} \colon V_s \to V_t$ for each $s \leq t$ such that $V_{r,r} = \id$ and
$V_{s,t} \circ V_{r,s} = V_{r,t}$ if $r \leq s \leq t$.
A \defn{morphism} $\varphi \colon V \to V'$ of persistent modules is a natural transformation.
Explicitly, it is an assignment of a linear map $\varphi_t \colon V_t \to V'_t$ for every $t \in \R$ making the diagram
\begin{equation*}
	\begin{tikzcd}
		V_{s} \arrow[r, "V_{s,t}"] \arrow[d] & V_{t} \arrow[d] \\
		W_{s} \arrow[r, "W_{s,t}"] & W_{t}
	\end{tikzcd}
\end{equation*}
commute for every pair $s \leq t$.

We denote abusively by $\kk$ and $0$ the functors that assigns $0$ and $\kk$ respectively to every object of $\R$ and the identity map to each morphism.

%Additionally, we remark that there is a canonical isomorphisms between chain complexes of persistence modules and \defn{persistence chain complexes}, i.e.,
%\[
%\Ch(\Fun(T, \Vec)) \cong \Fun(T, \Ch).
%\]
%\anibal{maybe unnecessary...}

\subsubsection{} Two persistence modules $V$ and $W$ are said to be \defn{$\delta$-interleaved} for some $\delta \geq 0$ if there are natural transformations $f \colon V_{\bullet} \to W_{\bullet + \delta}$ and $g \colon W_{\bullet} \to V_{\bullet + \delta}$ such that $f \circ g$ and $g \circ f$ are equal to the structure maps $W_{\bullet} \to W_{\bullet + 2\delta}$ and $V_{\bullet} \to V_{\bullet + 2\delta}$, respectively.
The \defn{interleaving distance} between them is defined by
\[
d(V, W) = \inf\set{\delta \geq 0 \mid V \text{ and } W \text{ are } \delta\text{-interleaved}}.
\]

\subsection{Barcodes and q-tameness}

Let $I$ be an interval in $\R$, i.e., $I \subseteq \R$ and, if $r \leq s \leq t$ with $r, t \in I$, then $s \in I$.
The persistence module $\kk[I]$, referred to as an \defn{interval module}, is defined explicitly by
\[
\kk[I]_t =
\begin{cases}
	\kk & \text{if } t \in I, \\
	\hfil 0 & \text{otherwise},
\end{cases}
\qquad \qquad
\kk[I]_{s, t} =
\begin{cases}
	\id & \text{if } s, t \in I, \\
	\hfil 0 & \text{otherwise}.
\end{cases}
\]

A persistence module $V$ is said to be \defn{interval decomposable} if there is a multiset of intervals $(I_\lambda)_{\lambda \in \Lambda}$ such that $V$ is isomorphic to $\bigoplus_{\lambda \in \Lambda} \kk[I_\lambda]$.
In this case we refer to the multiset as the \defn{barcode} of $V$ and denote it $\barc V$.
By Azumaya’s theorem \cite{azumaya1950theorem}, $\barc V$ is unique up to reordering.

Crawley-Boevey's result \cite{Crawley-Boevey.2015} is a widely used existence theorem for barcode decompositions.
It ensures that any persistence module \(V\) with \(V_t\) being a finite dimensional vector space for every \(t \in \R\) has a barcode decomposition.
However, this condition is often too restrictive.

A more general condition is q-tameness, please consult \cite{Chazal.2016a, Chazal.2016b} for the approach we summarize below.
A persistence module \(V\) is considered \defn{q-tame} if the rank of the map \(V_{s,t} \colon V_s \to V_t\) is finite for all \(s < t\).
For instance, the infinite product of interval modules \(\prod_{n \in \N_{> 0}} C([0,1/n))\) demonstrates that not all q-tame persistence modules can have a barcode decomposition in the traditional sense.
A persistence module \(V\) is termed \defn{ephemeral} if all the maps \(V_{s,t} \colon V_s \to V_t\) are zero for \(s < t\).
The \defn{radical} \(\rad V\) of a persistence module \(V\) is defined as the unique minimal submodule of \(V\) such that the cokernel of the inclusion \(\rad V \hookrightarrow V\) is an ephemeral persistence module.
Specifically, \((\rad V)_t = \sum_{s<t} \img V_{s,t}\).
For example, the radical of the infinite product \(\prod_{n \in \N_{> 0}} C([0, 1/n))\) is the direct sum \(\bigoplus_{n \in \N_{> 0}} C((0,1/n))\).
If \(V\) is q-tame its radical has a barcode decomposition, which describes the isomorphism type of \(V\) ``up to ephemerals".\footnote{
This idea is formalized using the so-called \textit{observable category}, which is equivalent to the quotient of the category of persistence modules by the subcategory of ephemeral persistence modules.
The barcode of the radical of a q-tame persistence module \(V\) serves as a complete invariant of \(V\) in the observable category.}
We define the barcode of a q-tame persistent module as the barcode of its radical.

\subsubsection{} Let $M$ and $M'$ be two possibly empty multisets of intervals.
A subset $P \subset M \times M'$ is said to be a \defn{partial matching} between $M$ and $M'$ if every interval $I \in M$ is matched with at most one interval of $M'$ and every interval $I' \in M'$ is matched with at most one interval of $M$.

The \defn{bottleneck distance} between $M$ and $M'$ is defined as
\[
d_B(M, M') = \
\inf \set[\big]{\mathrm{cost}(P) \mid P \subset M \times M' \text{ is a partial matching}}
\]
where $\mathrm{cost}(P)$ is the largest between
\[
\sup_{(I, I') \in P} \set[\Big]{\max\set{|a - a'|, |b - b'|} \text{ for } I = [a, b], I' = [a', b']}
\]
and
\[
\frac{1}{2} \sup \set[\Big]{|a - b| \text{ for } [a,b] \in M \cup M' \text{ unmatched}}.
\]

The interleaving distance of $q$-tame persistence modules agrees with the bottleneck distance of its barcodes.

\subsection{Persistent (co)homology}

\subsubsection{} An \defn{$\R$-diagram of spaces} $X$ is a functor from $\R$ to the category $\Top$ of (topological) spaces with (continuous) maps.
Explicitly, it consists of a space $X_r$ for each $r \in \R$ and a map $X_{s,t} \colon X_s \to X_t$ for each $s \leq t$ such that $X_{r,r} = \id$ and $X_{s,t} \circ X_{r,s} = X_{r,t}$ if $r \leq s \leq t$.

For any $k \geq 0$, applying the degree $k$ homology functor (with coefficients in $\kk$) to an $\R$-diagram of spaces produces the persistence module $\rH_k(X; \kk)$, where the morphisms are those induced by $\set{X_{r,s}}_{r \leq s}$.
We will refer to this persistence module as the \defn{persistent $k$-homology} of $X$.

It is also convenient to consider functors from $\R^\op$ to $\Vec$, which, by abuse of terminology, we also refer to as persistence modules.
The most important example is the \defn{persistent $k$-cohomology} $\rH^k(X; \kk)$ of an $\R$-diagram of spaces $X$.

An $\R$-diagram of spaces is said to be q-tame if its persistent homology and cohomology are q-tame as persistence modules over $\kk$.

%\subsubsection{} An important class of $\R$-diagrams is one-parameter filtrations of a space.
%These are $\R$-diagrams with $X_s$ being a subspace of $X_t$ if $s \leq t$.
%In this case, said space is $\colim X = \bigcup_r X_r$.
%
%Another natural construction of a persistence module from an $\R$-diagram of spaces is obtained by considering relative cohomology with $\kk$ coefficients $\rH^k(\colim X, X; \kk)$.
%Notice that this is a functor from $\R$ to $\Vec$, since cohomology is contravariant and $(\colim X, X_s) \supseteq (\colim X, X_t)$ for $s \leq t$.
%We refer to it as the \defn{persistent relative $k$-cohomology} of $X$.
%
%Another example, defined analogously to the previous two, is \defn{persistent relative $k$-homology} $\rH_k(\colim X, X; \kk)$.
%
%We say that a filtration $X$ is q-tame if its persistent homology is (for each degree and field coefficients).
%For example, filtrations that starts as the empty set and eventually stabilizes are q-tame.
%If the filtration is q-tame, then the four persistence modules above are q-tame and their barcodes can be related by canonical bijections \cite[Thm.~6.2]{bauer2023dualities}.\footnote{
%The correspondence between the homology and cohomology constructions is a consequence of the universal coefficient theorem and holds in general.
%To relate absolute and relative constructions, one also needs that the natural maps $\colim \rH_k(X_r) \to \rH_k(\colim X_r) \to \lim \rH_k(\colim X_r, X)$ be isomorphisms for every $k$.}