% !TEX root = ../vr_st.tex

\subsection{Persistent $\theta$-modules}

\subsubsection{}\label{ss:theta-modules}

A cohomology operation $\theta \in \cO(\kk,m; \kk,n)$ is \defn{linear} if \(\k\) is a field and the map \(\theta_{\cX} \colon \rH^m(\cX; \k) \to \rH^n(\cX; \k)\) is a linear transformation for any cellular space \(\cX\).
We denote the subset of $\cO(\kk,m; \kk,n)$ containing the linear cohomology operations by $\cO(m,n;\k)$ and omit the field \(\k\) from the notation when no confusion arises from doing so.
For any prime $p$, the field $\Fp$ is additively generated, so all cohomology operations in $\cA_p$ are linear.

A linear cohomology operation \(\theta \in \cO(\ell, m)\) induces a morphism of persistence modules, denoted \(\theta_X \colon \rH^\ell(X) \to \rH^m(X)\), for any cellular \(\R\)-space \(X\).
The \defn{image} and \defn{kernel} of $\theta_X$, denoted \defn{$\img_\theta(X)$} and \defn{$\ker_\theta(X)$} respectively, are the persistence modules
\begin{align*}
	\img_\theta(X)_r &= \img((\theta_X)_r)\,, &
	\ker_\theta(X)_r &= \ker((\theta_X)_r)\,, \\
	\img_\theta(X)_{s,t} &= \rH^m(X)_{s,t}\big|_{\img(\theta_s)}\,, &
	\ker_\theta(X)_{s,t} &= \rH^\ell(X)_{s,t}\big|_{\ker(\theta_s)}\,,
\end{align*}
where \(r, s, t \in \R\) and \(s \leq t\).

Since the identity and zero maps are linear cohomology operations, for any \(m \in \N\) we have
\[
\img_\id(X) \cong \ker_0(X) \cong \rH^m(X),
\]
which shows that these invariants are strictly more general than persistent cohomology.

\subsubsection{}\label{ss:theta-modules-q-tame}

If \(\rH^m(X)\) (resp. \(\rH^\ell(X)\)) is q-tame, then \(\img_\theta(X)\) (resp. \(\ker_\theta(X)\)) is also q-tame, implying that its barcode decomposition exists.
In this case we refer to its barcode as the \defn{$\img_\theta$-barcode} (resp. \defn{\(\ker_\theta\)-barcode}) of \(X\).
The Steenrod barcodes introduced in \cite{medina2022per_st} are specific instances of \(\img_\theta\)-barcodes, where \(\theta\) corresponds to a Steenrod square.
