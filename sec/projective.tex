% !TEX root = ../vr_st.tex

\section{Real projective spaces}

\subsection{Metric model}

For any integer $n \geq 1$ and real number $r > 0$, let $\bS^n(r)$ be the \defn{$n$-sphere} of radius $r$ centered at the origin of $\R^{n+1}$.
We consider it equipped with the geodesic distance $d$.

The \defn{$n$-real projective space} $\rp^n(r)$ is the quotient space $\bS^n(r)$ by the antipodal map $x \mapsto -x$ for all $x \in \bS^n$.
We equip $\rp^n(r)$ with the quotient metric:
\[
d\big([x],[x']\big) =
\min\set[\big]{d(x, x'), d(-x, x')}.
\]

To simplify notation we denote $\bS(1)$ by $\bS^n$ and $\rp^n(2)$ by $\rp^n$.
Notice that $\diam(\bS^n) = \diam(\rp^n) = \pi$.

The inclusion of \(\mathbb{S}^n\) into \(\mathbb{S}^{n+1}\) as the equatorial $n$-sphere is equivariant with respect to the antipodal action.
Therefore, it induces an inclusion of \(\rp^n\) into \(\rp^{n+1}\).

Since the \defn{infinite sphere} \(\mathbb{S}^\infty = \bigcup_n \mathbb{S}^n\) is contractible, its induced projection onto the \defn{real projective space} $\rp^\infty = \bigcup_n \rp^n$ defines its universal cover, so $\rp^\infty$ is a model for \(K(\Z/2, 1)\).

\subsection{Cohomology}

The cohomology algebra of $\rp^n$ with mod 2 coefficients is the polynomial algebra generated by a single element $\alpha$ in degree 1.
That is to say,
\[
\rH^\ast(\rp^\infty; \Z/2) \cong (\Z/2)[\alpha].
\]
Additionally, for any $n \in \N$,
\[
\rH^\ast(\rp^n; \Z/2) \cong \frac{(\Z/2)[\alpha]}{(\alpha^{n+1} = 1)}.
\]

The action of the Steenrod algebra $\cA_2$ on $\rH^*(\rp^n, \Z/2)$, for $n$ possibly equal to $\infty$, is either the 0 map or is given by
\[
\Sq^k(\alpha^m) = \binom{m}{k}\alpha^{m+k}
\]
for $0 \leq k \leq \frac{n-1}{2}$ and $k \leq m$.

\subsection{Filling Radii}

Recall the notion of filling radius, as discussed in \cref{ss:filling_radius}.
It is proved in \cite{katz1983filling} that for any $n \geq 1$
\[
\fillrad{\bS^n} = \frac{1}{2}\arccos\left(\frac{-1}{n+1}\right), \qquad
\fillrad{\rp^n} = \frac{\pi}{3}.
\]
We denote by $\zeta_n$ twice the filling radius of $\bS^n$, i.e., $\zeta_n = \arccos(\frac{-1}{n+1})$.

\subsection{Vietoris--Rips homotopy types}\label{prop:homotopy type}

In certain intervals, the homotopy type of the Vietoris--Rips complex of the $n$-sphere and $n$-projective space is known.
This information is presented in the following.

\label{prop:RPn}{\rm \cite[Thm.~4.5]{adams2022metric}.}
For $r \in (0,\frac{2\pi}{3} ]$
\[
\VR_r(\rp^n) \simeq \rp^n.
\]