% !TEX root = ../vr_st.tex

\section{Projective spaces}

\anibal{In this section we recall facts about projective spaces: real, complex and maybe quaternionic}

\subsection{Definitions} \

\subsubsection{Spheres} \
For any integer $n \geq 1$ and real number $r > 0$, let $\bS^n(r)$ be the $n$-sphere of radius $r$, equipped with the geodesic distance.

\subsubsection{Real projective spaces}
The \defn{$n$-real projective space} $\rp^n(r)$ is the quotient space $\bS^n(r)$ by the antipodal map $x \mapsto -x$ for all $x \in \bS^n$.
Let $[x]$ denote the equivalence class of $x$.
We equip $\rp^n(r)$ with the quotient metric defined as
\[
d_{\rp^n(r)}([x],[x']) =
\min\set[\big]{d_{\bS^n(r)}(x, x'), d_{\bS^n(r)}(-x, x')}.
\]
We write $\bS^n$ and $\rp^n$ for $\bS^n(1)$ and $\rp^n(2)$ respectively, so that
\[
\diam(\bS^n) = \diam(\rp^n) = \pi.
\] 

\subsection{Filling Radii}

It is proved in \cite{katz1983filling} that for any $n \geq 1$ their filling radius, as defined in \cref{ss:filling_radius}, are given by
\[
\fillrad{\bS^n} = \frac{1}{2}\arccos(\frac{-1}{n+1}), \qquad
\fillrad{\rp^n} = \frac{\pi}{3}.
\]
We denote by $\zeta_n$ twice the filling radius of $\bS^n$, i.e., $\zeta_n = \arccos(\frac{-1}{n+1})$.

\subsection{Cohomology rings}