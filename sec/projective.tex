% !TEX root = ../vr_st.tex

\section{Projective spaces}

\anibal{In this section we recall facts about projective spaces: real, complex and maybe quaternionic}

\subsection{Definitions}

\subsubsection{Spheres}\
For any integer $n \geq 1$ and real number $r > 0$, let $\bS^n(r)$ be the $n$-sphere of radius $r$, equipped with the geodesic distance.
We denote $\bS(1)$ simply by $\bS^n$ and notice that $\diam(\bS^n) = 1$.

\subsubsection{Real projective spaces}\
The \defn{$n$-real projective space} $\rp^n(r)$ is the quotient space $\bS^n(r)$ by the antipodal map $x \mapsto -x$ for all $x \in \bS^n$.
Let $[x]$ denote the equivalence class of $x$.
We equip $\rp^n(r)$ with the quotient metric defined as
\[
d_{\rp^n(r)}([x],[x']) =
\min\set[\big]{d_{\bS^n(r)}(x, x'), d_{\bS^n(r)}(-x, x')}.
\]
We denote $\rp^n(2)$ by $\rp^n$ and notice that $\diam(\rp^n) = \pi$.

\subsubsection{Complex projective spaces}\
The \defn{$n$-complex projective space} $\cp^n(r)$ is the quotient space $\bS^{2n+1}(r)$ by the \( \text{U}(1) \) action, which identifies points on the sphere up to multiplication by complex numbers of unit magnitude.
Let $[z]$ denote the equivalence class of $z$.
We equip $\cp^n(r)$ with the quotient metric defined as
\[
d_{\cp^n(r)}([z],[w]) =
\min\set[\big]{d_{\bS^{2n+1}(r)}(z, w), d_{\bS^{2n+1}(r)}(z, -w)}.
\]
We denote $\cp^n(2)$ by $\cp^n$ and notice that $\diam(\cp^n) = \pi$.

\subsection{Cohomology rings}

The antipodal map defines a free action of the group with two elements on the $n$-sphere \(\mathbb{S}^n\).
The quotient space under this action, denoted by \(\rp^n\), defines the \defn{real projective $n$-space}.
The inclusion of \(\mathbb{S}^n\) into \(\mathbb{S}^{n+1}\) as the equatorial $n$-sphere is equivariant with respect to the antipodal action.
Therefore, it induces an inclusion of \(\rp^n\) into \(\rp^{n+1}\).
Since the infinite sphere \(\mathbb{S}^\infty = \bigcup_n \mathbb{S}^n\) is contractible, its induced projection onto $\rp^\infty = \bigcup_n \rp^n$ defines the universal cover of \(\rp^\infty\), so the \defn{real projective space} $\rp^\infty$ is a model for \(K(\Z/2, 1)\).

%A first approximation to $\cO^\st(\Z/2)$, defined as $\colim_n \rH^{\ast + n}(K(\Z/2, n); \Z/2)$, is $\rH^\ast(K(\Z/2, 1); \Z/2)$ or, equivalently, $\rH^\ast(\rp^\infty; \Z/2)$.
Its cohomology algebra is the polynomial ring generated by a single element $\alpha$ in degree 1.
That is to say,
\[
\rH^\ast(\rp^\infty; \Z/2) \cong (\Z/2)[\alpha].
\]
For any $n \in \N$,
\[
\rH^\ast(\rp^n; \Z/2) \cong \frac{(\Z/2)[\alpha]}{(\alpha^{n+1} = 1)}.
\]
The action of $\cA_2$ on $\rH^*(\rp^n, \Z/2)$, for $n$ possibly equal to $\infty$, is either the 0 map or is given by
\[
\Sq^k(\alpha^m) = \binom{m}{k}\alpha^{m+k}
\]
for $0 \leq k \leq \frac{n-1}{2}$ and $k \leq m$.

\subsection{Filling Radii}

It is proved in \cite{katz1983filling} that for any $n \geq 1$ their filling radius, as defined in \cref{ss:filling_radius}, are given by
\[
\fillrad{\bS^n} = \frac{1}{2}\arccos(\frac{-1}{n+1}), \qquad
\fillrad{\rp^n} = \frac{\pi}{3}.
\]
We denote by $\zeta_n$ twice the filling radius of $\bS^n$, i.e., $\zeta_n = \arccos(\frac{-1}{n+1})$.

\subsection{Vietoris--Rips homotopy types}

\subsubsection{} In certain intervals, the homotopy type of the Vietoris--Rips complex of these spaces is known.
This information is presented in the following.

\medskip\proposition Let $n$ be a positive integer.
\begin{enumerate}[{\rm (a)}]
	\item\label{prop:S1}{\rm \cite[Thm.~7.4]{adamaszek2017vietoris}.}
	For $t \in (\frac{2n\pi}{2n+1}, \frac{2(n+1)\pi}{2n+3}]$
	\[
	\VR_t(\bS^1) \simeq \bS^{2n+1}.
	\]
	
	\item\label{prop:Sn}{\rm \cite[Thm.~10]{lim2020vietoris}.}
	For $t \in (0, \zeta_n]$
	\[
	\VR_t(\bS^n) \simeq \bS^n.
	\]
	
	\item\label{prop:RPn}{\rm \cite[Thm.~4.5]{adams2022metric}.}
	For $t \in (0,\frac{2\pi}{3} ]$
	\[
	\VR_t(\rp^n) \simeq \rp^n.
	\]
\end{enumerate}
