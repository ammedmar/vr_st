\documentclass{amsart}
\input{../aux/style}
% environments
\newtheorem{theorem}{Theorem}
\newtheorem*{theorem*}{Theorem}
\newtheorem{proposition}[theorem]{Proposition}
\newtheorem*{proposition*}{Proposition}
\newtheorem{lemma}[theorem]{Lemma}
\newtheorem*{lemma*}{Lemma}
\newtheorem{corollary}[theorem]{Corollary}
\newtheorem*{corollary*}{Corollary}
\theoremstyle{definition}
\newtheorem{definition}[theorem]{Definition}
\newtheorem*{definition*}{Definition}
\newtheorem{remark}[theorem]{Remark}
\newtheorem*{remark*}{Remark}
\newtheorem{example}[theorem]{Example}
\newtheorem*{example*}{Example}
\newtheorem{construction}[theorem]{Construction}
\newtheorem*{construction*}{Construction}
\newtheorem{convention}[theorem]{Convention}
\newtheorem*{convention*}{Convention}
\newtheorem{terminology}[theorem]{Terminology}
\newtheorem*{terminology*}{Terminology}
\newtheorem{notation}[theorem]{Notation}
\newtheorem*{notation*}{Notation}
\newtheorem{question}[theorem]{Question}
\newtheorem*{question*}{Question}

% hyphenation
\hyphenation{co-chain}
\hyphenation{co-chains}
\hyphenation{co-al-ge-bra}
\hyphenation{co-al-ge-bras}
\hyphenation{co-bound-ary}
\hyphenation{co-bound-aries}
\hyphenation{Func-to-rial-i-ty}
\hyphenation{colim-it}
\hyphenation{di-men-sional}

% basics
\DeclareMathOperator{\face}{d}
\DeclareMathOperator{\dege}{s}
\DeclareMathOperator{\bd}{\partial}
\DeclareMathOperator{\sign}{sign}
\newcommand{\ot}{\otimes}
\DeclareMathOperator{\EZ}{EZ}
\DeclareMathOperator{\AW}{AW}

% sets and spaces
\newcommand{\N}{\mathbb{N}}
\newcommand{\Z}{\mathbb{Z}}
\newcommand{\Q}{\mathbb{Q}}
\newcommand{\R}{\mathbb{R}}
\renewcommand{\k}{\Bbbk}
\newcommand{\sym}{\mathbb{S}}
\newcommand{\cyc}{\mathbb{C}}
\newcommand{\Ftwo}{{\mathbb{F}_2}}
\newcommand{\Fp}{{\mathbb{F}_p}}
\newcommand{\Cp}{{\cyc_p}}
\newcommand{\gsimplex}{\mathbb{\Delta}}
\newcommand{\gcube}{\mathbb{I}}

% categories
\newcommand{\Cat}{\mathsf{Cat}}
\newcommand{\Fun}{\mathsf{Fun}}
\newcommand{\Set}{\mathsf{Set}}
\newcommand{\Top}{\mathsf{Top}}
\newcommand{\CW}{\mathsf{CW}}
\newcommand{\Ch}{\mathsf{Ch}}
\newcommand{\simplex}{\triangle}
\newcommand{\sSet}{\mathsf{sSet}}
\newcommand{\cube}{\square}
\newcommand{\cSet}{\mathsf{cSet}}
\newcommand{\Alg}{\mathsf{Alg}}
\newcommand{\coAlg}{\mathsf{coAlg}}
\newcommand{\biAlg}{\mathsf{biAlg}}
\newcommand{\sGrp}{\mathsf{sGrp}}
\newcommand{\Mon}{\mathsf{Mon}}
\newcommand{\symMod}{\mathsf{Mod}_{\sym}}
\newcommand{\symBimod}{\mathsf{biMod}_{\sym}}
\newcommand{\operads}{\mathsf{Oper}}
\newcommand{\props}{\mathsf{Prop}}

% operators
\DeclareMathOperator{\free}{F}
\DeclareMathOperator{\forget}{U}
\DeclareMathOperator{\yoneda}{\mathcal{Y}}
\DeclareMathOperator{\Sing}{Sing}
\newcommand{\loops}{\Omega}
\DeclareMathOperator{\cobar}{\mathbf{\Omega}}
\DeclareMathOperator{\proj}{\pi}
\DeclareMathOperator{\incl}{\iota}
\DeclareMathOperator{\Sq}{Sq}
\DeclareMathOperator{\ind}{ind}

% chains
\DeclareMathOperator{\chains}{N}
\DeclareMathOperator{\cochains}{N^{\vee}}
\DeclareMathOperator{\gchains}{C}

% pair delimiters (mathtools)
\DeclarePairedDelimiter\bars{\lvert}{\rvert}
\DeclarePairedDelimiter\norm{\lVert}{\rVert}
\DeclarePairedDelimiter\angles{\langle}{\rangle}
\DeclarePairedDelimiter\set{\{}{\}}
\DeclarePairedDelimiter\ceil{\lceil}{\rceil}
\DeclarePairedDelimiter\floor{\lfloor}{\rfloor}

% other
\newcommand{\id}{\mathsf{id}}
\renewcommand{\th}{\mathrm{th}}
\newcommand{\op}{\mathrm{op}}
\DeclareMathOperator*{\colim}{colim}
\DeclareMathOperator{\coker}{coker}
\newcommand{\Hom}{\mathrm{Hom}}
\newcommand{\End}{\mathrm{End}}
\newcommand{\coEnd}{\mathrm{coEnd}}
\newcommand{\xla}[1]{\xleftarrow{#1}}
\newcommand{\xra}[1]{\xrightarrow{#1}}
\newcommand{\defeq}{\stackrel{\mathrm{def}}{=}}

% comments
\newcommand{\anibal}[1]{\noindent\textcolor{blue}{\underline{Anibal}: #1}}

% pdf
\newcommand{\pdfEinfty}{\texorpdfstring{${E_\infty}$}{E-infty}}

% mathrm
\newcommand{\rA}{\mathrm{A}}
\newcommand{\rB}{\mathrm{B}}
\newcommand{\rC}{\mathrm{C}}
\newcommand{\rD}{\mathrm{D}}
\newcommand{\rE}{\mathrm{E}}
\newcommand{\rF}{\mathrm{F}}
\newcommand{\rG}{\mathrm{G}}
\newcommand{\rH}{\mathrm{H}}
\newcommand{\rI}{\mathrm{I}}
\newcommand{\rJ}{\mathrm{J}}
\newcommand{\rK}{\mathrm{K}}
\newcommand{\rL}{\mathrm{L}}
\newcommand{\rM}{\mathrm{M}}
\newcommand{\rN}{\mathrm{N}}
\newcommand{\rO}{\mathrm{O}}
\newcommand{\rP}{\mathrm{P}}
\newcommand{\rQ}{\mathrm{Q}}
\newcommand{\rR}{\mathrm{R}}
\newcommand{\rS}{\mathrm{S}}
\newcommand{\rT}{\mathrm{T}}
\newcommand{\rU}{\mathrm{U}}
\newcommand{\rV}{\mathrm{V}}
\newcommand{\rW}{\mathrm{W}}
\newcommand{\rX}{\mathrm{X}}
\newcommand{\rY}{\mathrm{Y}}
\newcommand{\rZ}{\mathrm{Z}}
% mathcal
\newcommand{\cA}{\mathcal{A}}
\newcommand{\cB}{\mathcal{B}}
\newcommand{\cC}{\mathcal{C}}
\newcommand{\cD}{\mathcal{D}}
\newcommand{\cE}{\mathcal{E}}
\newcommand{\cF}{\mathcal{F}}
\newcommand{\cG}{\mathcal{G}}
\newcommand{\cH}{\mathcal{H}}
\newcommand{\cI}{\mathcal{I}}
\newcommand{\cJ}{\mathcal{J}}
\newcommand{\cK}{\mathcal{K}}
\newcommand{\cL}{\mathcal{L}}
\newcommand{\cM}{\mathcal{M}}
\newcommand{\cN}{\mathcal{N}}
\newcommand{\cO}{\mathcal{O}}
\newcommand{\cP}{\mathcal{P}}
\newcommand{\cQ}{\mathcal{Q}}
\newcommand{\cR}{\mathcal{R}}
\newcommand{\cS}{\mathcal{S}}
\newcommand{\cT}{\mathcal{T}}
\newcommand{\cU}{\mathcal{U}}
\newcommand{\cV}{\mathcal{V}}
\newcommand{\cW}{\mathcal{W}}
\newcommand{\cX}{\mathcal{X}}
\newcommand{\cY}{\mathcal{Y}}
\newcommand{\cZ}{\mathcal{Z}}
% mathsf
\newcommand{\sA}{\mathsf{A}}
\newcommand{\sB}{\mathsf{B}}
\newcommand{\sC}{\mathsf{C}}
\newcommand{\sD}{\mathsf{D}}
\newcommand{\sE}{\mathsf{E}}
\newcommand{\sF}{\mathsf{F}}
\newcommand{\sG}{\mathsf{G}}
\newcommand{\sH}{\mathsf{H}}
\newcommand{\sI}{\mathsf{I}}
\newcommand{\sJ}{\mathsf{J}}
\newcommand{\sK}{\mathsf{K}}
\newcommand{\sL}{\mathsf{L}}
\newcommand{\sM}{\mathsf{M}}
\newcommand{\sN}{\mathsf{N}}
\newcommand{\sO}{\mathsf{O}}
\newcommand{\sP}{\mathsf{P}}
\newcommand{\sQ}{\mathsf{Q}}
\newcommand{\sR}{\mathsf{R}}
\newcommand{\sS}{\mathsf{S}}
\newcommand{\sT}{\mathsf{T}}
\newcommand{\sU}{\mathsf{U}}
\newcommand{\sV}{\mathsf{V}}
\newcommand{\sW}{\mathsf{W}}
\newcommand{\sX}{\mathsf{X}}
\newcommand{\sY}{\mathsf{Y}}
\newcommand{\sZ}{\mathsf{Z}}
% mathbb
\newcommand{\bA}{\mathbb{A}}
\newcommand{\bB}{\mathbb{B}}
\newcommand{\bC}{\mathbb{C}}
\newcommand{\bD}{\mathbb{D}}
\newcommand{\bE}{\mathbb{E}}
\newcommand{\bF}{\mathbb{F}}
\newcommand{\bG}{\mathbb{G}}
\newcommand{\bH}{\mathbb{H}}
\newcommand{\bI}{\mathbb{I}}
\newcommand{\bJ}{\mathbb{J}}
\newcommand{\bK}{\mathbb{K}}
\newcommand{\bL}{\mathbb{L}}
\newcommand{\bM}{\mathbb{M}}
\newcommand{\bN}{\mathbb{N}}
\newcommand{\bO}{\mathbb{O}}
\newcommand{\bP}{\mathbb{P}}
\newcommand{\bQ}{\mathbb{Q}}
\newcommand{\bR}{\mathbb{R}}
\newcommand{\bS}{\mathbb{S}}
\newcommand{\bT}{\mathbb{T}}
\newcommand{\bU}{\mathbb{U}}
\newcommand{\bV}{\mathbb{V}}
\newcommand{\bW}{\mathbb{W}}
\newcommand{\bX}{\mathbb{X}}
\newcommand{\bY}{\mathbb{Y}}
\newcommand{\bZ}{\mathbb{Z}}
% mathfrak
\newcommand{\fA}{\mathfrak{A}}
\newcommand{\fB}{\mathfrak{B}}
\newcommand{\fC}{\mathfrak{C}}
\newcommand{\fD}{\mathfrak{D}}
\newcommand{\fE}{\mathfrak{E}}
\newcommand{\fF}{\mathfrak{F}}
\newcommand{\fG}{\mathfrak{G}}
\newcommand{\fH}{\mathfrak{H}}
\newcommand{\fI}{\mathfrak{I}}
\newcommand{\fJ}{\mathfrak{J}}
\newcommand{\fK}{\mathfrak{K}}
\newcommand{\fL}{\mathfrak{L}}
\newcommand{\fM}{\mathfrak{M}}
\newcommand{\fN}{\mathfrak{N}}
\newcommand{\fO}{\mathfrak{O}}
\newcommand{\fP}{\mathfrak{P}}
\newcommand{\fQ}{\mathfrak{Q}}
\newcommand{\fR}{\mathfrak{R}}
\newcommand{\fS}{\mathfrak{S}}
\newcommand{\fT}{\mathfrak{T}}
\newcommand{\fU}{\mathfrak{U}}
\newcommand{\fV}{\mathfrak{V}}
\newcommand{\fW}{\mathfrak{W}}
\newcommand{\fX}{\mathfrak{X}}
\newcommand{\fY}{\mathfrak{Y}}
\newcommand{\fZ}{\mathfrak{Z}}
\addbibresource{../aux/usualpapers.bib}

%%%%%%%%%%%%%%%%%%%%%%
% !TEX root = ../vr_st.tex

\DeclareMathOperator{\sus}{\Sigma}
\renewcommand{\Vec}{\mathsf{Vec}}
\newcommand{\bbR}{\mathbb{R}}
\newcommand{\RR}{\mathbb{R}}
\newcommand{\kk}{\mathbb{k}}
\newcommand{\st}{\mathrm{st}}

\definecolor{darkdgmcolor}{rgb}{0.0, 0.0, 0.8}
\definecolor{darkgreen}{rgb}{0.0, 0.8, 0.0}
\definecolor{purple}{RGB}{153,50,204}
\definecolor{dgmcolor}{RGB}{255,20,147}
\definecolor{barccolor}{RGB}{20,147,255}
\definecolor{myred}{RGB}{227,26,28}

\definecolor{darkblue}{rgb}{0,0,0.7} % darkblue color
\newcommand{\darkblue}{\color{darkblue}} % darkblue command
\newcommand{\defn}[1]{{\darkblue \emph{#1}}} % for defining new concepts

\newcommand{\ling}[1]{{ \textcolor{purple} {#1}}}
\newcommand{\facundo}[1]{{ \textcolor{blue} {#1}}}

\DeclareMathOperator{\rad}{rad}
\DeclareMathOperator{\VR}{VR}
\DeclareMathOperator{\opH}{H}
\newcommand{\PFD}{\mathrm{PFD}}
% \newcommand{\sqbarc}[1]{\Sq^{#1}\barc}

\DeclareMathOperator{\barc}{Bar}
\DeclareMathOperator{\img}{Im}

\newcommand{\Hpt}[4][\Ftwo]{\opH_{#2}(\VR_{#4}(#3);\,#1)}
\newcommand{\Hbarc}[3][\Ftwo]{\barc \opH_{#2}(\VR_\bullet(#3);\,#1)}
\newcommand{\sqbarc}[3][\Ftwo]{\barc\Sq^{#2}(\VR_\bullet(#3);\,#1)}

\newcommand{\fillrad}[1]{\operatorname{FillRad}(#1)}
\newcommand{\field}{\mathbb{F}}

\newcommand{\rp}{\mathbb{RP}}
\newcommand{\bbS}{\mathbb{S}} % for sphere
\newcommand{\diam}{\operatorname{diam}}

\newcommand{\db}{d_{\mathrm{B}}}
\newcommand{\dgh}{d_{\mathrm{GH}}}

\newcommand{\Coordinate}[2]%
{ \coordinate (#1) at (#2);
}

% Added by Ling
\newcommand{\thetabarc}[2][\Ftwo]{\barc\theta(\VR_\bullet(#2);\,#1)}
\newcommand{\thetamodule}[2][\k]{\img\left(\theta(#2;\,#1)\right)}
\newcommand{\dhi}{d_{\mathrm{HI}}}
\newcommand{\di}{d_{\mathrm{I}}}
\newcommand{\Xfunc}{X_\bullet}
\newcommand{\Yfunc}{Y_\bullet}
\newcommand{\Wfunc}{W_\bullet} % add commands here
\addbibresource{../aux/bibliography.bib} % add references here
\usepackage{enumitem}
\setlist{label=\arabic{enumi}.,itemsep=\medskipamount, left=0pt}

%%%%%%%%%%%%%%%%%%%%%%
\title[Referee reply]{\underline{REPLY TO REFEREE Y} \\ Persistent cohomology operations \\ and Gromov--Hausdorff estimates}

\newcommand{\ar}{\medskip\noindent\textit{Reply}:\ }
\renewcommand{\thesection}{\arabic{section}}

\begin{document}
	\noindent\today
	\maketitle

	We would like to thank the reviewer for a careful and insightful analysis of our paper, and for the many suggestions improving its presentation.
	We copy their report for completeness.

	\section{Reviewer's summary}

	Recent years have seen the development of the theory of persistent cohomology
	operations. The authors identify the lack of a comprehensive exploration of the
	properties of these operations in the literature and intend this paper to address
	this gap in the literature, as well as to contribute to questions of applicability of
	these operations.

	For a fixed linear cohomology operation $\theta$ and an $\mathbb{R}$-filtered space $X$, one can
	study the $\mathbb{R}$-filtered modules defined by the images $\operatorname{Im}\theta$ and kernels
	$\operatorname{Ker}\theta$ of $\theta$. This is what is understood by persistent cohomology
	operations. Relying on prior decomposition results for the Vietoris--Rips filtration
	$\mathrm{VR}(X)$ of a metric space $X$ on the level of spaces (namely, that
	$\mathrm{VR}(X \times Y) \simeq \mathrm{VR}(X) \times \mathrm{VR}(Y)$ and
	$\mathrm{VR}(X \vee Y) \simeq \mathrm{VR}(X) \vee \mathrm{VR}(Y)$), the authors establish analogous
	decomposition results for
	\[
	\operatorname{Im}^{\mathrm{VR}}_{\theta}(\cdot) := \operatorname{Im}\theta(\mathrm{VR}(\cdot))
	\quad \text{and} \quad
	\operatorname{Ker}^{\mathrm{VR}}_{\theta}(\cdot) := \operatorname{Ker}\theta(\mathrm{VR}(\cdot)).
	\]
	The authors also establish Vietoris--Rips stability results for persistent
	cohomology operations, and use lower bounds on the interleaving distance
	between persistence modules $\operatorname{Im}^{\mathrm{VR}}_{\mathrm{Sq}^k}$ of the wedge
	product of spheres $S^1 \vee \dots \vee S^n$ and the real projective space
	$\mathbb{RP}^n$ to establish new lower bounds for the Gromov--Hausdorff distance
	between these spaces (here $\mathrm{Sq}^k$ are the Steenrod squares). The authors
	demonstrate that the lower bounds achieved by using persistent cohomology
	operations are strictly greater than the bound possible from persistent homology,
	thus establishing a greater discriminatory power of persistent cohomology
	operations for this example. The bounds in the proofs rely on the knowledge of
	certain critical radii of round spheres $S^{n+1}(r)$ and of $\mathbb{RP}^n$; however,
	the methods in the proofs are more general and can be applied to more spaces
	if their critical radii are known (the intermediate theorems are stated for more
	general spaces for that purpose). For example, the authors specify the exact
	statements necessary to obtain lower bounds for the Lens spaces.

	The authors succeed in the expressed goals of giving a unified account of
	persistent cohomology operations, establishing theoretical properties (like
	decomposition results and stability results), and exhibiting applicability of
	these modules. The article is well-structured, clearly written, and provides the
	necessary background on cohomology operations, persistence functors, and
	metric geometry. It also provides comprehensive references to the current
	literature on these topics. Most proofs follow straightforward technical arguments,
	but require expert knowledge of previously established results in the above
	diverse topics, which makes a significant contribution to closing the existing
	gaps in the literature. Additionally, the article contributes to the study of
	Gromov--Hausdorff distances of metric spaces, which is a difficult topic with
	not many results known.

	\section{Reviewer's individual items}

	\begin{enumerate}
		\item P.~9, beginning of sec.~3.1.1, line ``We denote by $K(\pi, n)\dots$'':
		maybe note that $K(\pi, n)$ is connected.

		\ar Changed as suggested (CAS).

		\item In sec.~3.2.1, the cohomology operations are defined for cellular spaces.
		Why is such a restriction needed? Perhaps related, in the proof of Lemma~3.6.1,
		$Z$ is assumed to be cellular. I am not sure I see why it is needed in the proof.
		I am guessing this is because the authors want to cite results from~[BL23]
		(where the category of spaces is restricted to CGWH), but it would be
		nice if the authors could elaborate (add a couple of sentences, give specific
		references) on the necessity of the cellular assumption in general and for
		the proofs of Lemmas~3.6.1 and~3.6.2 in particular.

		\ar Together with the reasons presented by the referee, we also want to avoid the multiplicity of homology/cohomology theories that are available once the category of cellular spaces is abandoned.
		As the referee knows, there are multiple non-equivalent theories for more complicated spaces.

		\ling{I propose the following: \\
		\\
		Together with the reasons presented by the referee, we also want to avoid the multiplicity of homology/cohomology theories that are available once the category of cellular spaces is abandoned.
		We agree with the referee that it will be nice to elaborate on the necessity of the cellular assumption. We added the following sentences to the begining of Section 3: \\
		\\
		``Throughout this section we work with cellular (in particular CW) spaces.
		This ensures that singular and cellular (co)homology agree [Hat02, Theorem 2.35], allowing us to fix a single well-behaved cohomology theory.
		Moreover, since CW complexes are CGWH spaces, the results of~[BL23], formulated in the category of CGWH spaces, are directly applicable. This assumption is used in Lemmas~3.5.3, 3.6.1,
		and~3.6.2."
		}

		\item At the beginning of sec.~3.4.1, p.~12, the authors write
		``For any prime $p$, the field $\mathbb{F}_p$ is additively generated, so all
		cohomology operations in $\mathcal{A}_p$ are linear''. I am not sure it is very
		easy/straightforward to see the implication, so I think it’s better to also
		give a reference. (Alternatively, one could also state, with a reference,
		that stable operations are additive, or that the Steenrod squares are
		homomorphisms.)

		\ar We need to show that for any \(k \in \Fp\) the operation \(\phi\) satisfies \(\phi(k x) = k \phi(x)\) knowing that \(\phi(x+y) = \phi(x) + \phi(y)\).
		Since \(k x = \sum_{k} x\) this is immediate.

		\ling{I propose the following:\\
		\\
		We need to show that for any \(k \in \Fp\) the operation \(\phi\) satisfies \(\phi(k x) = k \phi(x)\) knowing that \(\phi(x+y) = \phi(x) + \phi(y)\).
		Since scalar multiplication in $\mathbb{F}_p$ is given by repeated addition (\(k x = \sum_{k} x\)), additivity implies $\mathbb{F}_p$-linearity.\\
		We have expanded the mentioned sentence to add the above explanation:\\
		\\
		``For any prime $p$, since scalar multiplication in $\mathbb{F}_p$ is given by repeated addition, additivity implies $\mathbb{F}_p$-linearity. Hence all cohomology operations in $\cA_p$ are linear.''
		}

		\item Sec.~3.5.3, p.~13, subscripts in $X_{s,y}$ and $M_{s,y}$ are probably
		$X_{s,t}$ and $M_{s,t}$.

		\ar Thank you, CAS.

		\item P.~18, sec.~4.3.1, the sentence ending in ``\dots since the thickening
		shares the same homology as the original Vietoris--Rips complex.''
		maybe needs a reference (e.g.~[Gil24], which is mentioned later in
		sec.~4.3.2).

		\ar Thank you, CAS. We have added `[Cor.~3, Gil24]' to the sentence.

		\item P.~21, sec.~4.4.3, in the proof of the Lemma, I am not sure I see why
		\[
		\sum_{b=2}^{a} d_S(\omega_b x_j^*, \omega_{b-1} x_j^*) < r(a-1),
		\]
		which seems to be needed. If the assumption is that
		$d_S(\omega_b x_j^*, \omega_{b-1} x_j^*) < r$, then it seems to contradict
		the last paragraph of the proof where $2\pi/q \le d_S(x_i^*, g x_i^*)$.
		I apologize if I am confusing something here.

		If I am not confusing something, and there is indeed a contradiction,
		then perhaps the proof can utilize the reverse triangle inequality with
		\[
		d_S(x_i^*, \omega_q x_j^*) \ge
		\left| d_S(x_i^*, x_j^*) - d_S(x_j^*, \omega_q x_j^*) \right|
		\]
		for any $q$ and the inequality
		\[
		d_S(x_i^*, x_j^*) \le d_S(x_i^*, x_0) + d_S(x_0, x_j^*) < 2r.
		\]

		\ling{I will work on this}

		\item P.~24, first paragraph, in the sentence
		``Moreover, this bar dies after $\delta_j$ as explained below.''
		shouldn’t it be ``dies before'' instead? Similarly, in the two sentences
		that follow it (``dies before'' instead of ``dies after'').
		Additionally, in this paragraph, one starts with the cycle representative
		with the lifespan $2\beta_{j-1}^i$ in $\mathrm{VR}$, so it seems to me that all
		death times have to have a coefficient $2$ (since one is working in
		$\mathrm{VR}_{2r}$, instead of $U_r$).

		\item P.~24, sec.~4.6, first paragraph, I think the reader might appreciate
		a reminder that here $S^n$ means $S^n(1)$. Similarly, in the proof of the
		Lemma in that section, it would be nice to have a reminder that
		$S^n_{C_q}$ is a projective/lens space of a different diameter.

		\ar Added ``the round sphere of radius 1'' to this sentence:
		We will consider every sphere \(\bS^n(2)\) to have radius 2 so the quotient \(\rp^n\) has the same diameter as \(\bS^n\), the round sphere of radius 1.

		Added ``Lens space'' to this sentence:
		We are interested in spheres \(\bS^{2n+1}(q)\) of radius \(q\), so the quotient Lens space \(\rL_q^n\) has the same diameter as the unit sphere \(\bS^{2n+1}\).

		\item Sec.~4.6, in the proof of the lemma, part~(2), sentence
		``By~[Kat83], the filling radius of $S^i(2)_{C_2}$ is $\pi/6$'':
		probably a typo and it should be $S^i(1)_{C_2}$.
	\end{enumerate}

	\section{Other changes}

	\begin{enumerate}
		\item
	\end{enumerate}
\end{document}